%%%%%%%%%%%%%%%%%
% This is an sample CV template created using altacv.cls
% (v1.3, 10 May 2020) written by LianTze Lim (liantze@gmail.com). Now compiles with pdfLaTeX, XeLaTeX and LuaLaTeX.
%
%% It may be distributed and/or modified under the
%% conditions of the LaTeX Project Public License, either version 1.3
%% of this license or (at your option) any later version.
%% The latest version of this license is in
%%    http://www.latex-project.org/lppl.txt
%% and version 1.3 or later is part of all distributions of LaTeX
%% version 2003/12/01 or later.
%%%%%%%%%%%%%%%%

%% If you need to pass whatever options to xcolor
\PassOptionsToPackage{dvipsnames}{xcolor}

%% If you are using \orcid or academicons
%% icons, make sure you have the academicons
%% option here, and compile with XeLaTeX
%% or LuaLaTeX.
% \documentclass[10pt,a4paper,academicons]{altacv}

%% Use the "normalphoto" option if you want a normal photo instead of cropped to a circle
\documentclass[10pt,a4paper,normalphoto]{altacv}

% \documentclass[10pt,a4paper,ragged2e,withhyper]{altacv}

%% AltaCV uses the fontawesome5 and academicons fonts
%% and packages.
%% See http://texdoc.net/pkg/fontawesome5 and http://texdoc.net/pkg/academicons for full list of symbols. You MUST compile with XeLaTeX or LuaLaTeX if you want to use academicons.

% Change the page layout if you need to
\geometry{left=1.25cm,right=1.25cm,top=1.5cm,bottom=1.5cm,columnsep=1.2cm}

% The paracol package lets you typeset columns of text in parallel
\usepackage{paracol}

% Change the font if you want to, depending on whether
% you're using pdflatex or xelatex/lualatex
\ifxetexorluatex
  % If using xelatex or lualatex:
  \setmainfont{Roboto Slab}
  \setsansfont{Lato}
  \renewcommand{\familydefault}{\sfdefault}
\else
  % If using pdflatex:
  \usepackage[rm]{roboto}
  \usepackage[defaultsans]{lato}
  % \usepackage{sourcesanspro}
  \renewcommand{\familydefault}{\sfdefault}
\fi

% Change the colours if you want to
\definecolor{SlateGrey}{HTML}{2E2E2E}
\definecolor{LightGrey}{HTML}{666666}
\definecolor{DarkPastelRed}{HTML}{450808}
\definecolor{PastelRed}{HTML}{8F0D0D}
\definecolor{GoldenEarth}{HTML}{E7D192}
\colorlet{name}{black}
\colorlet{tagline}{PastelRed}
\colorlet{heading}{DarkPastelRed}
\colorlet{headingrule}{GoldenEarth}
\colorlet{subheading}{PastelRed}
\colorlet{accent}{PastelRed}
\colorlet{emphasis}{SlateGrey}
\colorlet{body}{LightGrey}

% Change some fonts, if necessary
\renewcommand{\namefont}{\Huge\rmfamily\bfseries}
\renewcommand{\personalinfofont}{\footnotesize}
\renewcommand{\cvsectionfont}{\LARGE\rmfamily\bfseries}
\renewcommand{\cvsubsectionfont}{\large\bfseries}


% Change the bullets for itemize and rating marker
% for \cvskill if you want to
\renewcommand{\itemmarker}{{\small\textbullet}}
\renewcommand{\ratingmarker}{\faCircle}

%% sample.bib contains your publications
\addbibresource{sample.bib}

\begin{document}
\name{JOSE MENDOZA CHUPICAHUA}
\tagline{Ingeniería Geográfica}
%% You can add multiple photos on the left or right
\photoR{2.8cm}{FotoCarne}
% \photoL{2.5cm}{Yacht_High,Suitcase_High}

\personalinfo{%
  % Not all of these are required!
  \email{jose.mendoza41@unmsm.edu.pe}
  \phone{916194272 / 996379807}
  \mailaddress{Mz k Lote 9, Urb. Las Casuarinas, Puente Piedra}
  \location{Lima, Perú}
  \linkedin{https://www.linkedin.com/in/jose-mendoza41-unmsm-edu/}
  \github{josecMendoza41}
  %% You MUST add the academicons option to \documentclass, then compile with LuaLaTeX or XeLaTeX, if you want to use \orcid or other academicons commands.
  % \orcid{0000-0000-0000-0000}
  %% You can add your own arbtrary detail with
  %% \printinfo{symbol}{detail}[optional hyperlink prefix]
  % \printinfo{\faPaw}{Hey ho!}[https://example.com/]
  %% Or you can declare your own field with
  %% \NewInfoFiled{fieldname}{symbol}[optional hyperlink prefix] and use it:
  % \NewInfoField{gitlab}{\faGitlab}[https://gitlab.com/]
  % \gitlab{your_id}
}

\makecvheader
%% Depending on your tastes, you may want to make fonts of itemize environments slightly smaller
% \AtBeginEnvironment{itemize}{\small}

%% Set the left/right column width ratio to 6:4.
\columnratio{0.6}

% Start a 2-column paracol. Both the left and right columns will automatically
% break across pages if things get too long.
\begin{paracol}{2}
\cvsection{CONOCIMIENTOS}

\cvevent{AUTOCAD 2019}{UNIMASTER}{Marzo 2020 -- Abril 2020}{SMP, Lima}
\begin{itemize}
\item Intermedio
\item 32 Hrs
\end{itemize}

\divider

\cvevent{CURSO DE PILOTO A DISTANCIA RPAS}{Centro de Instruccion Aeronautica Civil CIAC N°004}{Enero 2020}{Magdalena del Mar}
\begin{itemize}
\item 20 Hrs
\end{itemize}

\divider

\cvevent{TOPOGRAFIA APLICADA A LA INGENIERIA}{ADO ENGINEERS}{Diciembre 2019 - En curso}{SMP, Lima}
\begin{itemize}
\item Estacion Total, Geodesia, Fotogrametria, Autocad Civil
\item 90 Hrs
\end{itemize}

\divider

\cvevent{PROCESAMIENTO DE IMAGENES SATELITALES}{GEOGIS INGENIEROS S.A}{Agosto 2018 -- Setiembre 2018}{SMP, Lima}
\begin{itemize}
\item Envi 5.3
\item Intermedio
\item 32 Hrs
\end{itemize}

\divider

\cvevent{TELEDETECCION USANDO R}{SRsat}{Diciembre 2018 -- Enero 2019}{Virtual}
\begin{itemize}
\item 21 Hrs
\end{itemize}

\divider

\cvevent{ARCGIS 10.3}{GEOGIS INGENIEROS S.A}{Enero 2018 -- Abril 2018}{SMP, Lima}
\begin{itemize}
\item Avanzado
\item 48 Hrs
\end{itemize}

\divider

\cvevent{FOTOGRAMETRIA CON RPAS PHANTOM 4 PRO - RTK Y EBEE SENSEFLY}{LINELCON S.A.C}{Enero 2020 -- Marzo 2020}{Arequipa, Perú}
\begin{itemize}
\item Procesamiento en Agisopht Photoscan y Pix4D
\item Modalidad Virtual
\end{itemize}

\divider

\cvevent{TELEDETECCION CON PYTHON}{CIECI}{Marzo 2019 -- Abril 2019}{Virtual}
\begin{itemize}
\item 40 Hrs
\end{itemize}

\divider

\cvevent{Monitoreo de la Perturbación de Bosques usando Senores Remotos}{CIECI}{Marzo 2019 -- Abril 2019}{Virtual}
\begin{itemize}
\item 40 Hrs
\end{itemize}

\cvsection{FOROS Y TALLERES}

\cvevent{Conferencia}{GESTIÓN DE RESIDUOS SÓLIDOS, SIG Y EL CAMBIO CLIMÁTICO Y ELABORACION DE MAPAS BASE Y TEMATICOS A PARTIR DE IMÁGENES SATELITALES Y MODELOS DE ELEVACIÓN DIGITAL(DEM).
}{ABRIL 2016}{UNMSM}

\divider

\cvevent{Congreso}{ I CONGRESO INTERNACIONAL DE INGENIERIA GEOGRÁFICA}{SETIEMBRE 2018}{CIP}
\begin{itemize}
\item Colegio de Ingenieros del Perú
\end{itemize}

\divider

\cvevent{Taller}{INTRODUCCION A LA TELEDETECCIÓN POR RADAR}{NOVIEMBRE 2018}{UNMSM}
\begin{itemize}
\item CIECI
\end{itemize}

\divider

\cvevent{Taller}{PYTHON APLICADO A RECURSOS HIDRICOS}{NOVIEMBRE 2018}{UNMSM}
\begin{itemize}
\item CEIGA
\end{itemize}

\medskip

%\cvsection{A Day of My Life}

% Adapted from @Jake's answer from http://tex.stackexchange.com/a/82729/226
% \wheelchart{outer radius}{inner radius}{
% comma-separated list of value/text width/color/detail}
%\wheelchart{1.5cm}{0.5cm}{%
%  6/8em/accent!30/{Sleep,\\beautiful sleep},
%  3/8em/accent!40/Hopeful novelist by night,
%  8/8em/accent!60/Daytime job,
%  2/10em/accent/Sports and relaxation,
%  5/6em/accent!20/Spending time with family
%}

% use ONLY \newpage if you want to force a page break for
% ONLY the current column
%%\newpage

%%\cvsection{Publications}

%%\nocite{*}

%%\printbibliography[heading=pubtype,title={\printinfo{\faBook}{Books}},type=book]

%%\divider

%%\printbibliography[heading=pubtype,title={\printinfo{\faFile*[regular]}{Journal Articles}},type=article]

%%\divider

%%\printbibliography[heading=pubtype,title={\printinfo{\faUsers}{Conference Proceedings}},type=inproceedings]

%% Switch to the right column. This will now automatically move to the second
%% page if the content is too long.
\switchcolumn

\cvsection{DESCRIPCIÓN}

\begin{quote}
``Estudiante de Ingeniería Geográfica de la Universidad Nacional Mayor de San Marcos, Noveno Ciclo. Mención en Geomática y Ordenamiento Territoral.''
\end{quote}

\cvsection{EXPERIENCIA}

\cvachievement{\faBriefcase}{CIMA}{Apoyo en la Toma de datos Geofísicos}

\divider

\cvachievement{\faBriefcase}{INPE}{Georreferenciación y Representación cartográfica de la Población Penitenciaria.}

\cvsection{INFORMÁTICA}

\cvtag{ARCGIS}
\cvtag{ENVI}
\cvtag{AGISOPHT}\\
\cvtag{AUTOCAD}
\cvtag{QGIS}
\cvtag{Pix4D}
\cvtag{Python Basic}
\cvtag{R Basic}
\cvtag{TBC}\\
\cvtag{Civil 3D}

\cvsection{Idiomas}

\cvskill{Inglés}{2}
\divider

\cvskill{Español}{5}
\divider

%% Yeah I didn't spend too much time making all the
%% spacing consistent... sorry. Use \smallskip, \medskip,
%% \bigskip, \vpsace etc to make ajustments.
\medskip

\cvsection{EDUCACION}

\cvevent{Ingeniería Geográfica}{Universidad Nacional Mayor de San Marcos}{Marzo 2016 -- En Curso}{}


\divider

\cvevent{Secundaria}{I.E.I Estados Unidos }{Marzo 2009 -- Diciembre 2013}{}

% \divider

\cvsection{Referencias}

% \cvref{name}{email}{mailing address}
\cvref{Instituto Nacional Penitenciario}{Unidad de Estadistica}{arivera@inpe.gob.pe \\   mlujand@inpe.gob.pe}
{ 940 701 408}




\end{paracol}


\end{document}
